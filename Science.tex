% !TEX root = UCase.tex
% \vskip -1em
\onecolumngrid
% \begin{center}
% %{\bf{\em \large Unconventional superconductivity in the layered intermetallic YFe$_2$Ge$_2$}}\\
% %{\bf{\em \large Unconventional superconducting and normal states in transition metal compounds -- a new generation of ultra-pure crystals}}\\
% %{\bf{\em \large Origins of mass renormalisation and unconventional superconductivity in transition metal compounds}}
% %{\bf {\em \large Superconducting and normal states in quantum materials  \\}}
% %Unconventional superconducting and normal states \\ in transition metal compounds\\}}
% %\vspace{2em}
% { Part 2: Case for support}
% \vspace{-1em}
% \end{center}

\begin{mdframed}[hidealllines=true,backgroundcolor=blue!5,innerleftmargin=5pt,innerrightmargin=5pt,innertopmargin=5pt,innerbottommargin=5pt]
\noindent
New materials are the lifeblood of condensed matter physics. This project combines wide-ranging experimental, computational and theoretical studies into the origins of superconductivity and the associated anomalous normal states in four novel material systems. % that represent a broad spectrum of electronic and vibrational energy scales.
%that include (i) the new iron-based superconductor YFe$_2$Ge$_2$, now available at unprecedented purity levels that exceed the best crystals grown outside Cambridge by at least an order of magnitude, (ii) the quantum critical heavy fermion superconductor CeNi$_2$Ge$_2$, for the first time now available as high purity single crystals suitable for quantum oscillation studies, (iii) the newly discovered unconventional high pressure superconductor CeSb$_2$, which shows ultra-strong electronic correlations and an unusually high critical field that far exceeds the paramagnetic limit, (iv) quasi-periodic host-guest structures such as high pressure Sb-II and Ba-IV, which feature low-lying sliding modes that facilitate strong-coupling superconductivity. 
Our findings will refine the criteria guiding the search for new superconductors, increasingly  targeted for  practical applications. % that represent a broad spectrum of electronic and vibrational energy scales.
\end{mdframed}
% \vspace{1em}
%   \includegraphics[width=\columnwidth]{Figures/SuperconResurgent2}
  
\twocolumngrid

%Note: include references to potential referees:  Sebastian, Balakrishnan, \\

\subsection*{Vision: systematic search and investigation of new unconventional superconductors}
\noindent
\paragraph{Superconductivity resurgent.} Superconductivity research is ramping up globally, driven by (i) the recognition that superconductors facilitate large-volume applications for instance in fusion research, accelerators, MRI scanners, generators and motors, and power distribution, as well as device applications in computing and sensing; (ii) exciting breakthroughs in fundamental research across different material systems ranging from the cuprates and Fe-based high temperature superconductors to organics, twisted bilayer graphene and f-electron systems; (iii) materials breakthroughs, including the ability to induce near-room temperature superconductivity %under extreme pressure 
in supercompressed superhydrides \cite{drozdov19,somayazulu19}, the discovery of \SI{80}{\kelvin} superconductivity in a pressurised, novel nickelate \cite{sun23}, and the discovery of multiple field resilient superconducting states in CeRh$_2$As$_2$ \cite{khim21} and UTe$_2$ \cite{aoki19,ran19}. 


% One of the most exciting recent developments in condensed matter research has been the
% demonstration of \myul{superconductivity in superhydrides} near room temperature but at very high pressure
% \cite{snider20,drozdov19,somayazulu19}. 
% The compressed superhydrides demonstrate the power of engineering a phonon-mediated superconducting pairing mechanism towards optimal outcomes. Further gains are possible by widening the scope towards 
%In compressed superhydrides such as LaH\textsubscript{10}, superconductivity is mediated by dynamic deformations of
%the crystal lattice which, because of the light mass of the hydrogen
%atoms and the high spring constants caused by strong compression, reach
%to very high frequencies. This approach could produce
%high temperature superconductivity at ambient pressure, if -- like diamond -- a metastable high pressure structure could be brought back to ambient conditions. %This is not the case for LaH$_{10}$. 

In conventional superconductors, the pairing interaction is communicated by lattice vibrations. Fundamental and applied superconductivity research are increasingly examining \myul{unconventional superconductors}, which instead harness the {strong electronic interactions} that are also responsible for magnetism and that are
known in some cases to reach coupling strengths equivalent to several
thousand Kelvin \cite{monthoux07,norman11}. Like rare minerals that occur in seams, these superconductors are thinly spread across the space of all accessible materials but richly concentrated within those families on which most current research is focused, which include, for example, various copper oxide, iron or cerium compounds. 

%Unconventional superconductivity in cuprates, for instance, has already produced transition temperatures $T_c$ approaching $160~\text{K}$. To optimise the effects of direct
% electronic interactions further requires a better understanding of the underlying and often competing
% mechanisms in these complex materials. 
%Medium-term commercial impact will arise from new materials identified as a consequence of our research. Our experimental results will help refine numerical models that will guide the search for new superconductors with high volume applications, including those mentioned in the introduction above. 

%Only a few material families are so far known to exhibit superconductivity that is not mediated by lattice deformations, or phonons, alone \cite{monthoux07,norman11}. 

\paragraph{Uranium-based superconductors} make up a large fraction of the overall still limited number of unconventional superconductors (Table). This material family is highly diverse in terms of crystal and electronic structure. %Superconductivity in U compounds is often found near the threshold of ferromagnetism, but many U superconductors are antiferromagnetic or appear far from the threshold of magnetism. 
Studying and learning from these U-based superconductors can accelerate the wider search for unconventional superconductors with desirable properties, but the bewildering diversity and complexity of phenomena and materials challenges in this class of materials renders a detailed and comprehensive understanding at best time-consuming and difficult. 

\begin{figure}
  \includegraphics[width=\columnwidth]{Figures/UTe2Overview.pdf}
  \caption{{\bf Key properties of UTe$_2$.} (a) Crystal structure and Fermi surface geometry deduced from quantum oscillation measurements, showing compensated electron and hole pockets of constant cross-section that undulate slightly along the $c$-axis. (b) Temperature dependence of the superconducting upper critical field $B_{c2}$ along the $b$-axis, showing transition between two superconducting states, SC1 and SC2. (c) Angle dependence of the low temperature $B_{c2}$, showing the resilience of SC2 to applied fields reaching up to a metamagnetic transition at $\simeq \SI{35}{\tesla}$ and a third superconducting state SC3 that reaches to far higher field values, still. Our improved 'MSF' crystals reach higher critical fields in the SC1 state (dark blue) and superconductivity extends over a wider angle range in the SC2 state (pink) than previous 'CVT' generations of samples.}
  \label{fig:UTe2}
\end{figure}

\paragraph{Examining the new superconductor UTe$_2$} can produce important insights for understanding unconventional superconductors more generally: (i) it holds at least three, probably more, distinct superconducting states, which can be selected by varying applied field, temperature and pressure, (ii) at least some of these are triplet pairing states, as demonstrated for instance by NMR measurements and by the unusual resilience of superconductivity to applied field of up to $\simeq \SI{60}{\tesla}$ in certain field directions, (iii) low temperature magnetic order identified at moderate pressure and strong magnetic fluctuations observed by neutron scattering at ambient pressure strongly suggest a central role for a magnetic pairing mechanism, a strong contender also in other unconventional superconductors such as the Ce- and Yb-based heavy fermion systems and Fe- or Cu-based high temperature superconductors.

\paragraph{Clean crystals for a clearer view.} Until recently, work on UTe$_2$ was hampered by poor crystal quality. Long electronic mean free paths are required to establish anisotropic forms of order, such as unconventional superconductivity. Previous generations of samples exhibited a residual Sommerfeld ratio $C/T$ in the low-$T$ limit within the superconducting state, which caused much theoretical speculation that could quickly be dismissed once $C/T \rightarrow 0$ could be established in cleaner crystals more recently. Likewise, informative investigation methods such as quantum oscillation measurements require high quality samples and proved impossible for several years, causing ample speculation about the electronic structure near the Fermi energy.  \myul{New growth methods} pioneered by our project partners at Charles University, Prague, now produce pristine single crystals with superior quality and electronic mean free path as measured by their residual resistance ratio (RRR) of order 500, an order of magnitude improvement on previous best efforts. Using this new generation of ultra-clean crystals grown using the molten salt flux (MSF) technique, we were able to detect quantum oscillations with unprecedented clarity, enabling us to resolve the Fermi surface structure of UTe$_2$ \cite{eaton23}. %These high quality crystals with residual resistivity ratio (RRR) of order 500 display a significantly enhanced $T_c$ compared to previous generations of samples. 

\paragraph{Simple Fermi surface.} Because disorder is always relevant in unconventional superconductors, many initial findings in UTe$_2$, need to be re-examined in these new crystals. We have already found that the superconducting critical fields are significantly enhanced, whereas the metamagnetic transition remains unchanged \cite{wu23}. Moreover, our recent quantum oscillation and quasiparticle interference oscillation measurements \cite{eaton23,weinberger23} have revealed a surprisingly simple Fermi surface geometry, which consists of just two compensated, cylindrical, slightly corrugated pockets (\autoref{fig:UTe2}), populated by heavy quasiparticles. The accuracy and simplicity of this result is reminiscent of the case of Sr$_2$RuO$_4$ \cite{bergemann03}. It presents a solid point of departure for modelling the magnetic or charge response functions in UTe$_2$ and ultimately for understanding in detail the nature and origin of its superconducting states.

\paragraph{UTe$_2$ now presents a clean model system,} because of the pristine quality of newly available single crystals and the simple Fermi surface geometry. Our primary vision for this project is to investigate those key aspects of UTe$_2$ that -- like the electronic structure --  reveal underlying principles and can lead towards a working model for superconductivity and magnetism in this intriguing material. These involve, in particular, careful mapping out of the high pressure, field and temperature phase diagram, the determination of the magnetic order induced at high pressure, and the investigation of normal state properties and Fermi surface by multi-probe studies and quantum oscillation measurements at high pressure. As the project unfolds we will turn this methodology also to related U-based superconductors such as UGe$_2$. 

%These new, cleaner crystals offer the opportunity to clarify many of the open scientific questions surrounding this interesting but complex material.

%and our group has played a leading role in discovering several such examples \cite{mathur98,saxena00,grosche00,zou14}.
% We urgently need to find new unconventional superconductors: 
% %they present proving grounds for advanced computational techniques, generate fresh impulses that advance our understanding of superconducting quantum materials in general and thereby prepare the ground for the next discovery,  helping us formulate guiding principles -- heuristic filters needed to select promising candidate systems. 
% not only are they scientifically interesting -- with every case  studied, the guiding principles for finding new superconducting material families can be refined.
% An example of such a guiding principle is illustrated in \autoref{fig:Guiding}, namely to home in on the threshold of magnetic order. There, at a so-called  \myul{quantum phase transition}, magnetic excitations reach to low energies. They mediate a long-ranged interaction which can stabilise superconductivity with an unconventional order parameter structure %(e.g. $s_\pm$, $p$- or $d$-wave)  
% \cite{monthoux07}. Such non-phononic pairing interactions are strongly tuneable. This causes superconducting domes 
% %as in Fig.~{\ref{fig:Guiding} 
% which in some cases are surprisingly narrow, explaining why this type of superconductivity is often found not by random searches but by scanning phase diagrams systematically near the border of magnetism. % -- for which high pressure studies are well suited.


\begin{figure*}
  \includegraphics[width=1.9\columnwidth]{Figures/PhaseDias.pdf}
  \caption{{\bf Template phase diagrams guiding materials exploration:}  (a) High pressure phase diagram of CePd$_2$Si$_2$, showing a superconducting region attached to the 
  threshold of antiferromagnetism \protect\citesel{mathur98}. Considering the added effect of magnetic field adds a third dimension:
  in UGe$_2$ (b), 
  superconductivity appears within a ferromagnetic region, which itself branches into two metamagnetic sheets \protect\cite{kotegawa11}. In PrPtAl or NbFe$_2$ \protect\citesel{friedemann18} (c), by contrast, ferromagnetism is replaced by an antiferromagnetic or spin-density wave region. *** Use URhGe rather than UGe$_2$ for middle panel ***
  %With hindsight, the  observation that iron-based superconductivity often peaks near the disappearance of magnetic order, whether achieved
  %by doping or pressure, resonates with the earlier realisation that low-\emph{T}\textsubscript{c} Ce-based
  %superconductors are frequently found on the threshold of magnetism. 
}
  \label{fig:Guiding}
  
\end{figure*}
  
   
\subsection*{Research questions}
\noindent
Our project will address the following \myul{key research questions} of wider relevance in superconducting quantum materials:

% These four material families span a broad range of electronic energy scales, from the low Kelvin regime in high pressure CeSb$_2$ to thousands of Kelvin in Bi, Sb and Ba, with YFe$_2$Ge$_2$ %($\sim \SI{400}{\kelvin}$) 
% and CeNi$_2$Ge$_2$  %($\sim \SI{100}{\kelvin}$) 
% connecting the two extremes. Nevertheless, they share striking similarities, such as tunability by pressure or composition and proximity to magnetic or structural instabilities. 
% Above their superconducting transitions, they all exhibit anomalous `normal' states with a sub-quadratic temperature dependence of the electrical resistivity, signalling low-lying excitations. In Bi-III (\autoref{fig:Materials}d), this non-Fermi liquid form of the resistivity may be attributed to low-lying sliding \ul{phonon} modes, which are a built-in consequence of its quasiperiodic structure. %This illustrates the central role of soft modes for causing anomalous electronic transport properties and also for understanding strong-coupling superconductivity. 
% By contrast, \ul{magnetic} excitations that become long-lived and long-range near the threshold of magnetism likely play the dominant role in the other three material families. 

%Exploiting these contrasts and similarities, 


\paragraph{a) Superconducting states:} UTe$_2$ hosts at least three distinct superconducting states, which can be selected in applied field. NMR data \cite{} and the magnitude of the upper critical field $B_{c2}$ strongly suggest that at least some of these involve triplet Cooper pair states. This produces a rare opportunity to study the nature and origin of triplet superconductivity in a clean system: what are the spatial order parameter wavefunctions (a large number of candidates are allowed by symmetry, but experiments can rule in or rule out candidates). Why is triplet superconductivity favoured? What causes high critical fields, in particular if they far exceed the paramagnetic limit (UTe$_2$ SC2 and SC3 states) or if superconductivity becomes reentrant (UTe$_2$ SC3)? %as in high-pressure CeSb$_2$? 
What is the origin of residual $C/T$ in the low $T$ limit in less pure samples of UTe$_2$, and can this be quantitatively attributed to impurity bound states? What is the nature of the vortex lattice in the mixed state, and does it change as we tune UTe$_2$ between the different superconducting states SC1-3? Are there further superconducting states accessible under pressure, as preliminary studies suggest \cite{}, and what is the role of spin-orbit coupling in determining the supercnoducting state?

Addressing these questions provides new insights for applications in other material families: what determines $B_{c2}$, and what material properties can help maximise it? How do we identify anisotropic order parameters experimentally, how can they be manipulated, how do we find materials that host them?

\paragraph{b) Correlated `normal' state:} % out of which the superconductivity emerges:
superconductivity in UTe$_2$ and other U-based systems arises out of a strongly correlated normal state. What is the origin of the high electronic heat capacity and enhanced quasiparticle mass in UTe$_2$ and other U-based heavy fermion compounds, which typically exceeds density functional theory (DFT) values by at least an order of magnitude? 
%$C/T$ is intermediate between heavy d-metal compounds (e.g. KFe2As2, YFe2Ge2) and Ce-compounds. Although there will be parallels with 4{\em f}-electron systems such as the Ce- and Yb-based heavy fermion compounds, 
Many U-based superconductors have more than one {\emph f}-electrons on each U site. This introduces strong local correlations via \myul{Hund's coupling}, which may produce significant baseline mass renormalisation, as demonstrated, for instance, in our recent QO study of YFe$_2$Ge$_2$ \cite{baglo22a}. What is the {\emph f}-electron count in UTe$_2$, does it change with applied pressure, and is the mass renormalisation uniform across the Fermi surface, supporting the Hund's metal scenario outlined above? Do normal state transport and thermodynamic properties universally follow Fermi liquid theory, or can we reach -- under applied field or pressure -- (quantum critical) regions in the phase diagram in which the electrical resistivity takes a sub-quadratic temperature dependence or the heat capacity Sommerfeld ratio $C/T$ does not saturate to a constant at low $T$? 
%Analysing normal state properties is crucial, because  %How does the form of the effective interaction connect to microscopic models such as the Hubbard model for correlated metals near Mott localisation, the Kondo lattice model for 4$f$-electron heavy fermion superconductors, or the Hund's metal in some of the Fe-based superconductors \cite{georges13}? 
%What is the number of itinerant f electrons?  Age-old problem of e.g. UPt3 QO



\paragraph{c) Magnetic, charge, and 'hidden' order:} superconducting and normal state properties are controlled by the effective interaction between charge carriers, or quasiparticles. In contrast to the bare Coulomb interaction, the effective interaction can be dynamic, it can couple to spin, and it can be tuned by varying underlying material parameters. In many currently known unconventional superconductors, the interaction is predominantly magnetic \cite{monthoux07}, but different mechanisms are possible. These might involve density, valence, quadrupolar or orbital degrees of freedom, individually or in combination, which may be assessed by investigating ordered states nearby. In UTe$_2$, magnetic order is induced by moderate applied pressure $\simeq \SI{17}{\kilo\bar}$, but what is the precise nature of this magnetic order, and how is it affected by applied field and pressure? Because the pairing interaction can be mediated by magnetic fluctuations \cite{monthoux07}, where are the regions in the \myul{pressure-field-temperature phase diagram} (see \autoref{fig:Guiding} for examples), where this interaction can become long-ranged? Moreover, \myul{charge density wave} order (CDW) may be wide-spread in UTe$_2$ and other unconventional superconductors. What is the nature of CDW order, how does it affect superconductivity, and can we tune it by applied pressure or strain? In at least one prominent U-based superconductor, URu$_2$Si$_2$, a thermodynamic phase transition into a still unidentified 'hidden order' state continues to defy resolution. Is such hidden order more wide-spread across U-compounds, and what is its nature and origin?
FM qcp -> 1st order or SDW or something else altogether (URu2Si2). 
Lessons from NbFe2, PrPtAl. Fermi surface instabilities
Central e.g. to UGe2 story. Maybe happens more generally?! Local vs. band, orbitally selective Mott transitions? 





%Can we explain not only why some materials superconduct but also why other, very similar materials do not? What factors determine the variation of $T_c$ within the same material family? 

%Superconducting properties will be investigated in multi-probe studies with a wide range of collaborating groups (see below), in some cases in combination with applied pressure.



%\paragraph{c) Nature and tunability of effective interaction:}
%In the latter,  the electrons in the far more extended Fe $d$-states, which are forced into a high spin state, introduce strong local correlations that may be seen as analogous to Kondo lattice physics. Is this conceptual connection between rare-earth based heavy fermion compounds and the newer Fe-based superconductors helpful in understanding both material families?
%Can the Kondo energy be tuned to zero in special cases, as proposed in scenarios of orbitally selective Mott transitions or Kondo breakdown (e.g. \cite{si01,paul07}), and what are the consequences of ultra-low Kondo scales for superconducting and normal state properties? %What is the relationship between the underlying crystalline, electronic and magnetic structure and the form of the effective quasiparticle interaction? 
% Can we understand and control the energy scales that enter these microscopic models,  and can we exploit their tunability to vary superconducting and normal state properties? 
%At a microscopic level, rare-earth based, so-called heavy fermion superconductors are often discussed in a Kondo-lattice model in terms of the interplay between electrons in extended $s$, $p$ and $d$-states, which form broad metallic energy bands, and more localised electrons in tightly confined $f$-states, which take on properties of local magnetic moments at high temperatures or high energies. 
%A crucial parameter in this picture is the Kondo energy scale, which sets an effective bandwidth in the low temperature limit.
%How can aspects of the effective interaction be varied, for instance by changing composition or pressure, to affect normal state properties, the superconducting transition temperature $T_c$ or critical fields?   

% More ideas: \\
% \begin{itemize}
%  \item Most early hf superconductors were U-based. Maybe because it's actually wide-spread in U compounds? 

%\item $C/T$ is intermediate between heavy d-metal compounds (e.g. KFe2As2, YFe2Ge2) and Ce-compounds

% \item
% Local vs. band, orbitally selective Mott transitions? What is the number of itinerant f electrons? 
% Age-old problem of e.g. UPt3 QO

% \item 
% On-site correlations, Hund's metal 
% How do we actually pick that up experimentally?

% \item
% Role of Spin-orbit coupling?

% \item 
% FM qcp -> 1st order or SDW or something else altogether (URu2Si2). 
% Lessons from NbFe2, PrPtAl

% \item 
% What determines Bc2, and what can we learn from it? 
% And are there ways to maximise Bc2?

% \item 
% Are there vortex state transitions?
% Maybe lessons to be learned for applications

% \item 
% Multicomponent order parameters
% How do we identify them? How are they manipulated? How can we find new materials that host them?

% \item 
% Fermi surface instabilities
% Central e.g. to UGe2 story. Maybe happens more generally?!

%\item Nature of hidden order states, as in URu$_2$Si$_2$

% \end{itemize}



\subsection*{Approach: \\ multi-probe studies in clean crystals}
%\subsection*{Programme and Methodology}
\noindent
The research programme capitalises on our recent breakthroughs in UTe$_2$ research \cite{eaton23,wu23,weinberger23} and our long track record of discovery research in heavy fermion superconductors such as CePd$_2$Si$_2$ and CeIn$_3$ \cite{mathur98}, UGe$_2$ \cite{saxena00}, CeCu$_2$Si$_2$ \cite{yuan02}, and CeSb$_2$ \cite{squire23}. %related materials UAu$_2$ \cite{oneill22} and UGe$_2$ \cite{saxena00}. We will build our ongoing UTe$_2$ studies into a programme of multi-probe experiments across applied field, pressure and strain.

The planned experiments (\autoref{tab:Methods}) exploit our expertise and facilities in scanning tunneling spectroscopy (St. Andrews) and high precision transport, magnetic, thermodynamic and ultrasound measurements under extreme conditions (Cambridge and Edinburgh) of hydrostatic pressure (piston-cylinder and anvil cell devices, reaching up to $\SI{>100}{\kilo\bar}$), magnetic field (up to $\SI{20.4}{\tesla}$ in the lab, with higher fields available at international facilities) and low temperature (down to $\SI{<0.03}{\kelvin}$). We will continue to refine and extend experimental methods, with particular emphasis on high pressure temperature modulation calorimetry, ultrasound and quantum oscillation measurements  \citesel{squire23,semeniuk23}.  %In-house studies will focus on:

The programme is structured into four work packages (WP). WP1 maps out the rich phase diagram of UTe$_2$ in field, temperature, pressure and strain. WP2 addresses the need to resolve and understand the low temperature ordered states, magnetic, superconducting or otherwise. Quantum oscillation experiments probing the electronic Fermi surface play a central role in this effort, but will be supplemented by numerous complementary probes such as tunneling spectroscopy, transport studies, ARPES etc. Where possible experiments will extend to high pressure to investigate properties in those regions of the phase diagram that are of particular interest. The crucial WP3 concerns the growth and characterisation of clean crystals of our candidate materials as well as increasingly the exploration for new materials of interest, and WP4 covers the development of new instrumentation underpinning all of our studies. 


\begin{table}
  \centerline{\includegraphics[width=0.7\columnwidth]{Figures/MethodsTable.pdf}}
  \caption{{\bf Techniques} to be used in this project. In the column 'Who', C=Cambridge, E=Edinburgh, A=St. Andrews. The columns show which sample environments (applied field, hydrostatic pressure or uniaxial strain) can be combined with each measurement technique. }
  \label{tab:Methods}
\end{table}

\subsubsection*{Work package 1 (WP1): mapping \\
the phase diagrams} 
% a. Rho, C under pressure: Cambridge + Edinburgh as needed
% b. Ultrasound: Edinburgh
% c. MuSR: Alex
% d. Magnetisation under pressure: Montu
\noindent The power of mapping out magnetic field, pressure and composition phase diagrams is illustrated in \autoref{fig:Guiding}. The recent example of UTe$_2$ demonstrates that unexpected twists such as the ultra-high field superconductivity SC3, resilient up to \SI{60}{\tesla} for a narrow range of field orientations, could easily be missed without careful examination of a material's phase diagram over wide parameter ranges. Producing and refining the multi-dimensional phase diagram of an unconventional superconductor like UTe$_2$ provides the point of departure for more probing studies focusing on critical regions of the phase diagram and moreover by itself produces important clues for the nature of the underlying low temperature states. For instance, the surprising resilience of the superconducting states to applied field already points towards a triplet character of the superconducting order parameter, motivating follow-up by specialised probes. More generally, shape of the pressure, temperature, field phase diagram is of overarching importance, because it allows us to identify quantum critical points, where a pairing interaction could be expected to become relevant, and it would furthermore allow us to track the evolution of the enigmatic SC3 and SC2 superconducting pockets.
% We will examine the role of magnetic quantum phase transitions by joint pressure and composition tuning within the space spanned 
We will use transport (electrical resistivity, Hall effect), thermodynamic (heat capacity), magnetic (magnetisation, muSR) and structural (XRD, ultrasound) techniques at applied pressure or strain and in applied fields to survey pressure/strain/field/temperature phase diagrams and thereby  (i) delineate distinct magnetic, CDW or superconducting states and (ii) locate critical regions where response functions are expected to peak, e.g. near quantum phase transitions.



\objective{Map out field, temperature, pressure and strain phase diagrams in high purity samples of UTe$_2$ to correlate superconducting and normal state properties with magnetic quantum phase transitions.
\objective{Widen these studies to other U-based superconductors, in particular UAu$_2$ and UGe$_2$.}
%and thereby to
%correlate normal and superconducting state properties with changes in
%the electronic structure.
\\}


%\paragraph{Objectives in WP2, normal state properties}
\subsubsection*{WP2: resolving \\
the low temperature states}
\noindent
Having identified regions of interest, the associated low temperature states -- superconducting, magnetic, charge density wave -- and also the properties of the underlying normal state need to be resolved. 

\paragraph{Superconducting states:} the nature of the superconducting order parameter can be inferred from a combination of bulk and surface probes. The former include measurements of the temperature dependence of heat capacity, thermal conductivity, nuclear magnetic resonance, or ultrasound attenuation. Of particular importance is the temperature dependence of penetration-depth, which can be determined using the tunnel-diode oscillator technique (Edinburgh, with Bristol) and via careful analysis of muon spin rotation studies.
A key technique available in this project is scanning tunneling spectroscopy (STS) in the vicinity of surface defects, which has proven to be successful in effectively imaging the gap geometry \cite{}. At least for superconducting states accessible in low fields, we will apply this approach to the new generation of ultra-clean UTe$_2$ samples. 

Further information can be inferred from the response of the superconducting states to an artificially reduced electronic mean free path in the bulk. Electron irradiation offers a highly controlled approach for varying defect concentration. By tracking the response of the three distinct superconducting states in UTe$_2$ to increasing defect concentration, we can detect changes in the superconducting order parameter.
% combining a wide range of specialised experimental techniques (listed above)  will help resolve the gap structures in YFe$_2$Ge$_2$, LuFe$_2$Ge$_2$, CeNi$_2$Ge$_2$ and high-pressure CeSb$_2$. 
Analysis will incorporate the role of impurity bound states, which for a sign-changing gap produce distinct signatures in  all low $T$ properties, by numerical studies as in \cite{bang17} and by varying the impurity level.
% a. Neutron diffraction: zero-pressure -> Edinburgh, high pressure -> Montu
% b. XRD: Edinburgh single crystal XRD
% c. TDO penetration depth: Edinburgh with Bristol
% d. Tunneling: Edinburgh with St. Andrews
% e. MuSR: Alex
% f. XMCD: Montu



\paragraph{Magnetic order}, which in UTe$_2$ is induced at high pressure $>\SI{17}{\kilo\bar}$, will be investigated by high pressure neutron diffraction, muon spin rotation or X-ray magnetic circular dichroism, as well as by high pressure magnetisation measurements.  to probe magnetic or superconducting ground state properties. 
%These largely facilities-based experiments will be complemented by laboratory-based studies such as scanning tunneling spectroscopy (St. Andrews) and 
% Background
\paragraph{Fermiology:}
key input for any
theoretical description derives from the observation of quantum oscillations in high magnetic fields, a precise signature of the electronic Fermi surface and
carrier mass. Quantum oscillations have already been observed by us and others in UTe$_2$ at ambient pressure, but important questions are still unresolved. Moreover, it will be important to track the evolution of the Fermi surface and carrier mass as the magnetically ordered state is approached and crossed with pressure. We have pioneered rf tunnel-diode based techniques for tracking quantum oscillations in anvil pressure cells, which we will apply now to UTe$_2$ and, later, to other U-based superconductors. Ambient pressure quantum oscillation measurements in UTe$_2$ and other materials of interest will be referenced against information derived from STS studies (see also above).
Ambient pressure and high pressure quantum oscillation surveys on the Cambridge 20.4 Tesla/dilution refrigerator cryomagnet will be augmented by measurements up to 30 Tesla at the new all-superconducting facility in Beijing, where we have recently carried out preliminary measurements, up to 37 Tesla at the HFML Nijmegen facility or higher fields, still, at NHMFL Tallahassee. 
%In YFe$_2$Ge$_2$ and CeNi$_2$Ge$_2$, we
%have already observed 
%de Haas-van Alphen quantum
%oscillations, . 
%The fact that quantum oscillations could be observed despite non-Fermi liquid signatures in resistivity data presents a paradox which invites closer examination. 
 
%High pressure quantum oscillation surveys will also be used to investigate the unusual normal state properties in CeSb$_2$ and in quasiperiodic materials.
WP n Quantum oscillation studies 

 

Understanding the electronic properties of the normal state out of which unconventional superconductivity emerges is crucial for subsequently interrogating the microscopic character of the superconducting state(s). In our recent (ambient pressure) high-field quantum oscillation (QO) measurements at NHMFL, Florida, USA we resolved the Fermi surface (FS) of UTe2 [cite Eaton et al]. Unlike several other uranium-based superconductors the UTe2 FS is remarkably simple, consisting of just two cylindrical sheets (see fig X). This simplicity proffers the tantalising possibility that an accurate microscopic description of UTe2 may soon be within our grasp. 

 

A key outstanding question concerns the evolution of the FS under pressure - particularly as the putative QCP at $p_c \simeq \SI{17}{\kilo\bar}$ is crossed. A magnetically ordered state at $p > p_c$ has been observed, which is proposed to be antiferromagnetic in nature but is yet to be conclusively determined (see muSR section).  Mapping the FS through $p_c$ will therefore provide vital clues pertaining to the role of magnetic and quantum critical fluctuations in driving the various exotic superconducting phases.

Tracking QOs under pressure is a challenging problem, but one for which we are ideally placed to tackle. Contactless RF resistivity measurements will be performed using our methodology recently deployed in [cite Semeniuk PNAS]. At ambient pressure we recently made the surprising observation of quantum interference oscillations (QIOs) in UTe2 [cite Weinberger et al], in addition to 'conventional' QOs from Landau quantisation. These provide information about the k-space gaps between neighbouring Fermi sheets along with the difference in the effective carrier masses; recently we have also observed QIOs in preliminary measurements under pressure. By combining detailed QO and QIO measurements as a function of pressure we will build a full map of both the Fermi surface geometry and the Fermi velocity distribution, yielding valuable insight as to the hybridisation of the Te p-bands with the U d- and f-bands. To our knowledge no such synergistic combinatory investigation – of both QOs and QIOs simultaneously – has previously been performed on any heavy fermion compounds. This novel approach may therefore be of particular interest in a range of other materials, with this study forming the groundwork for future QO+QIO investigations to directly resolve the FS geometry and fermi velocity distribution. 

 

 

WP n+1 High magnetic field measurements 

 

The phase diagram of UTe2 - as revealed by prior measurements on CVT specimens – is remarkably rich, with intertwined charge- and pair-density wave orderings coexisting with the groundstate (SC1) superconductivity. Under applied magnetic fields two further superconducting states emerge (SC2 \& SC3) with SC3 spectacularly extending to ~ 70 T [cite Ran Nat Phys \& Helm et al], the highest ever recorded upper critical field for a re-entrant superconducting phase, or indeed for any heavy fermion superconducting state. In the new generation of crystals our preliminary measurements have already revealed that this rich phase landscape is markedly different, with the angular domain of SC2 showing an acute sensitivity to crystalline quality [cite Wu et al, see fig. X]. This naturally calls for a detailed mapping of the SC3 state in MSF samples, requiring measurements at pulsed magnetic field facilities for which we have recently been granted time. 

 

Furthermore, under pressure four distinct superconducting states have been identified [cite Aoki review], with complex field-angle dependencies. Understanding how these phases evolve as a function of pressure and magnetic field tilt angle is challenging, as the axial length of typical pressure cells greatly restricts their ability to be rotated in the narrow bores of resistive magnets. Thus to date measurements have only been performed on separate samples with individual field orientations, with phase diagrams then stitched together, which greatly complicates the understanding of this remarkably complex phase landscape. 

 

To address this, we have recently designed a miniaturised piston cylinder cell [see fig X] in collaboration with NHMFL, Florida, USA, specifically designed to fit into their recently constructed wide bore series-connected 36 T hybrid magnet. This will allow us to pioneer the combination of applied pressure and magnetic field tilt angle, enabling the full UTe2 phase diagram to be mapped on a single sample in-situ for the first time, thereby eliminating uncertainty as to how this plethora of exotic emergent phases evolves under pressure. This wide-bore magnet is also compatible with the Razorbill strain cell [see section X], which will allow us to pioneer the combination of high field and uniaxial pressure as well - which will yield important insights pertaining to the superconducting order parameter of the SC2 state [maybe cite \& discuss Ramshaw/write about strain separately?] 

 

 

 

\paragraph{Non Fermi liquid signatures} will be examined using high-precision thermodynamic and transport measurements across pressure, magnetic field and temperature, in order to pin down the regions in the phase diagram where they extend to lowest temperature and correlate them with quantum critical phenomena arising from nearby ordered states. The role of disorder will be examined in samples of varying purity levels. 


\objective{Probe the superconducting states in UTe$_2$, UAu$_2$ and UGe$_2$ with complementary techniques in order to resolve the superconducting order parameter structure.\\}

% \objective{Resolve magnetic, electronic and vibrational excitations by neutron scattering, ARPES, ultra-high field quantum oscillation measurements and Raman spectroscopy.\\}
\objective{Develop a theoretical understanding of superconductivity and of anomalous normal state properties in all four material systems.}
\objective{Resolve the Fermi surface and carrier mass and its evolution with pressure in UTe$_2$ by quantum oscillation surveys, extending later to related materials.\\}

\objective{Survey non-Fermi liquid signatures using high precision temperature sweeps into the milli-Kelvin range, in fields up to $\SI{20}{\tesla}$ and pressures up to $\SI{100}{\kilo\bar}$.\\}

\vspace{ -1.0em}

\subsubsection*{WP3: crystal growth and materials discovery}
\noindent
a. MSF 
b. Induction furnace
c. CVT

\highlight{Availability of depleted uranium for crystal growth}
%The discoveries summarised in \autoref{fig:Materials} build on our group's recent progress in high-purity crystal growth and high pressure techniques. In this work package, we will employ and refine these techniques and conduct systematic searches for new superconducting quantum materials. 
%\paragraph{New ultra-pure crystals generate new opportunities.}
Crystal quality plays a central role in the discovery of new collective phenomena in quantum materials. 




\begin{figure}[t]
  %  \centerline{\includegraphics[width=0.95\columnwidth]{Figures/CompareKFA-YFG3b}}
  \centerline{\includegraphics[width=\columnwidth]{Figures/Structures.pdf}}
  
     \caption{{\bf Material families} identified for the first stage of this project. As the programme unfolds, the investigation will widen first to other U-based superconductors (see table) and eventually to $d$-metal compounds with strong Hund's coupling and hence underlying similarity to the headline materials. }
     
      \label{fig:Materials}
  \end{figure}
  
 
  
\begin{table}
  \begin{tabular}{l L{5.5cm}}
    UAu$_2$ &
    AFM, s/c under pressure, A. Huxley 21, 22 PNAS, possible multi-component order parameter, QO under pressure?
    \\

    UBe$_{13}$ & \\
    
    UCoGe & 
    FM \\
    
    U$_6$Fe & 
    Perhaps CDW at 110K? Whitley PhD. Looks like interesting linear-in-T rho(T) but little good transport data published. RRR ~ 10 \\
    
    UGe$_2$  & FM \\
    UIr & \\
    UPd$_2$Al$_3$ & 
    AFM, TN 14K, Tc 2K
    \\

    UNi$_2$Al$_3$ &
    AFM, TN 1.4 K, Tc 1 K
    \\

    UPt$_3$ & \\
  
    URhGe & 
    FM
    \\

    URu$_2$Si$_2$ & 
    Hidden order \\
  
    UTe$_2$ & 
      dHvA under pressure
      Strain in high field, to investigate multi-component SC1/SC2
      Pioneer miniaturised pressure cells – map full p-B-T-angle phase diagram 
      QIOs under pressure, combined with dHvA gives full 3D details of FS without needing to rotate \\
  \end{tabular}
\end{table}

  \subsection*{Selected materials}
  \noindent
  % Driven by this overarching objective, 
  For these reasons, our project will initially investigate three material systems available as \myul{high-quality single crystals}, in which discoveries and enabling breakthroughs %in crystal growth and high pressure techniques 
  have occurred as recently as last summer (\autoref{fig:Materials}):
  %Recent discoveries and advances in crystal growth and high pressure techniques enable a broad study across four material systems, which 
  %The chosen materials display common phenomena but access a wide range of electronic or vibrational energy scales  (\autoref{fig:Materials}):
  %the new \uline{iron-based superconductors} YFe$_2$Ge$_2$ and LuFe$_2$Ge$_2$, the \uline{moderate heavy-fermion superconductors} CeNi$_2$Ge$_2$ and CePd$_2$Si$_2$, the \uline{ultra-heavy fermion superconductor} CeSb$_2$ and the \uline{quasiperiodic superconductors} high pressure  Bi, Sb and Ba.
  %We will investigate four superconducting material systems  
  
  \paragraph {a) UTe$_2$:} 
  %including the new system YFe$_2$Ge$_2$ and its relative LuFe$_2$Ge$_2$, which straddle an antiferromagnetic quantum phase transition and exhibit  an unusually high heat capacity $C$ at low temperature $T$ ($C/T \simeq 100~\text{mJ/molK}^2$), consistent with our observation of carrier mass renormalisation among the highest recorded in transition metal compounds \citesel{baglo21}. 
  % Whereas LuFe$_2$Ge$_2$ is antiferromagnetic below about $\SI{6}{\kelvin}$, its isoelectronic sister compound YFe$_2$Ge$_2$ is paramagnetic. 
  
%   WP crystal growth of UTe2
% Background

Although synthesising single crystals of UTe2 has proved to be straightforward via chemical vapour transport progress in improving sample quality has required the systematic optimization of parameters (Cairns J. Phys.: Condens. Matter 32 (2020) 415602, Rosa et al Nature Commun Mater 3, 33 (2022)). As the ratio of Te/U in the deposition zone (controlled by the reagent composition and temperature) grows towards 2:1, Tc of the resultant crystals increases, but then falls abruptly. The dramatic drop in Tc is attributed to U deficiency. As well as its role in controlling the composition in CVT a lower growth temperature also appears to lower the intrinsic concentration of U vacancies formed. Growth from molten salt (NaCl/KCl) flux occurs at lower temperature than achieved with CVT (with iodine as transport agent) and has resulted in higher Tc and RRR crystals. The molten salt grown crystals are however smaller (sub mm) and have a large scatter in quality within a single batch. The current crystals grown in Edinburgh by CVT have sharp heat capacity transitions above 2.0 K, but with varying non-SC fractions. While the thermodynamics of the processes involved in the growth process are straightforward to quantify, the kinetics is equally important and is not well characterised.
Methodology.

Both CVT and molten salt growth will be carried out. We will characterise our crystals alongside those form our collaborators (CU-Prague) to ensure we have the best material available for subsequent study. Characterisation will  by inhouse Laue, Energy dispersive X-ray (including FIB), PPMS (heat capacity and susceptibility to 300K - 350mK, 7 Tesla) and homebuilt heat capacity (to 4K - 20 mK 14 Tesla) as well as transport.

We aim to better optimise the synthesis process. We will do this by incorporating fibre optics into our CVT furnace to image the growth process in real time. This will give a better understanding of the kinetics. For the molten salt growth as well as following known recipes with a standard vertical furnace we will apply RF heating directly to the carbon crucible to better control the temperature gradient at the conical growth tip to maximize crystal yield and size.

The work programme will combine optimizing growth to produce the best samples possible and making 

  
  % discovered superconductivity in YFe$_2$Ge$_2$ \citesel{zou14,chen16,chen19}, emerging out of an anomalous normal state with a %non-Fermi liquid 
  % $T^{3/2}$ power-law dependence of the resistivity. 
  % %The non-Fermi liquid $T^{3/2}$ form of the electrical resistivity, the pronounced effect of disorder scattering on superconductivity \cite{chen19} and theoretical studies \cite{singh14,subedi14} indicate that YFe$_2$Ge$_2$ is an unconventional superconductor. 
  % More recently, we also found superconductivity in the newest generation of high purity crystals of LuFe$_2$Ge$_2$ (\autoref{fig:Materials}a). 
  % These findings depended on our ability to produce ultra-high quality crystals of both materials, with purity levels in YFe$_2$Ge$_2$ exceeding those of the best samples grown outside Cambridge tenfold \citesel{chen20b}. %This new generation of crystals will enable wide-ranging multi-probe studies in both materials as well as non-superconducting reference compounds, which will be outlined below.
 
  
% \paragraph{Crystal growth:} UTe$_2$, UAu$_2$ and UGe$_2$ will be grown using our carefully optimised .. .   We will further improve our growth protocols ... 

%The growth programme will widen to include other Fe-based intermetallics such as LaFe$_2$Ge$_2$, YFe$_2$Si$_2$, and CaFe$_2$Ge$_2$ as well as their composition series with YFe$_2$Ge$_2$. 
When flux growth or chemical vapour transport are not productive, we use cold-crucible %polycrystals will be
induction melting, and we will
explore Czochralski and Bridgman growth for single crystal production. We will continue to improve these techniques by using higher quality starting materials, by tuning the growth protocol and by optimising the annealing procedure.  

%\vspace{0.5em}
\noindent 
We will widen our  programme
\begin{leftlist}
\item UAu$_2$ and UGe$_2$
\highlight{Text about UAu$_2$ and UGe$_2$} 

\item to other well-known U-based heavy fermion superconductors such as UPt$_3$ and URu$_2$Si$_2$. With recent advances in instrumentation a re-examination of the superconducting and magnetic states in the former is becoming timely. Little is known, moreover, regarding its evolution with pressure and strain. In the latter, which hosts an enigmatic hidden order state below about \SI{17}{\kelvin} at ambient pressure, quantum oscillation measurements to higher magnetic fields than were possible in the past will reveal much-needed information about Fermi surface geometry and carrier mass. 
%Ce-based Kondo-lattice systems such as CePd$_2$Si$_2$ (\autoref{fig:Guiding}), the ferromagnet CeAgSb$_2$ \cite{logg13}, of which we have recently grown crystals with $RRR>180$ and the antiferromagnet CeAl$_2$.
%Both CeAgSb$_2$  and CeAl$_2$ have critical pressures of about $\SI{35}{\kilo\bar}$, now well within the range of our high pressure measurements.
 %We will continue to improve this technique by using higher quality starting materials, by tuning the growth protocol and by optimising the annealing procedure. 
\item to related U-based systems such as U$_6$Fe or UBe$_{13}$. %Fe-based intermetallics such as LaFe$_2$Ge$_2$, YFe$_2$Si$_2$, and CaFe$_2$Ge$_2$ as well as their composition series with YFe$_2$Ge$_2$, and 
\item to other materials: we expect the expertise gained in the growth of UTe2 to also enhance capability to grow other U -Te.  There are several interesting 2D van der Waals magnetic systems with higher Te content that could be synthesised. These may also give insight into UTe2 and appear as defects.

\item $d-$metal compounds that may mimic some of the properties of the U-based superconductors which form the central objective of this project.
 % of current interest, including the high pressure superconductor MnP \cite{cheng15} and the Kagom\'e lattice  superconductors (K/Rb/Cs)V$_3$Sb$_5$ \cite{ortiz21}, which have already been grown in our lab, as well as the ruthenate high pressure superconductor Ca$_2$RuO$_4$ \cite{alireza10}. %\hl{note also Nowotny phase work}
\end{leftlist}


\paragraph {Sample characterisation} will involve powder and
single-crystal x-ray diffraction as well as electron microprobe analysis,
and the determination of magnetic, thermodynamic and transport
properties using our dedicated
SQUID magnetometer and PPMS (both with $^3$He inserts).  
As part of WP2, more detailed investigation of the nature of disorder and impurities will be carried out in collaboration with ... using high resolution single crystal x-ray diffraction and electron microscopy.
d. Characterisation: transport/thermodynamic/magnetic; XRD; TEM


\paragraph{Materials discovery:}
we will follow up fresh opportunities in targeted searches for altogether new unconventional superconductors. %Systematic studies on the material systems already at the centre of the proposal will inspire targeted searches: 
% For instance, can we expect to hit a quantum critical point in high pressure studies of YFe$_2$Si$_2$ or LaFe$_2$Ge$_2$, mentioned above? Can we extend insights from Fe-based systems to Mn, Ni, Co or Ru-based materials? Can we find relatives to high-pressure CeSb$_2$? 
\myul{Pressure-assisted high throughput surveys} play a central role in these searches, as in previous discoveries (e.g. Figs.~\ref{fig:Materials}b-d). % by leveraging the power of computation.
Further acceleration is possible  by more accurate selection of candidate materials, for which we will increasingly complement heuristic filters by numerical calculations with collaborators (see also WP2).  


%Further to seeking out materials with nearby magnetic order, other filter criteria could be (i) moderately enhanced low temperature heat capacity, (ii) high room temperature resistivity, (iii) layered structure, and (iv) potential for making pure samples.




%\paragraph{Objectives in WP1, Crystal growth}
\objective{Further improve the quality of UTe$_2$, UAu$_2$ and UTe$_2$ crystals by studying the origins of disorder in these material systems, and grow superior crystals for studies of superconducting and normal states. Grow related systems and substitution series to map out composition phase diagrams.}\\
%\objective{Grow high-quality crystals of Tl$_2$Ba$_2$CuO$_{6+\delta}$ and HgBa$_2$CuO$_{4+\delta}$ for measurements exploring the phase diagram and the evolution of electronic properties.}\\
\objective{Explore new superconducting quantum materials in pressure-assisted high-throughput surveys guided by heuristic and -- increasingly -- computational filters (also WP 2).}\\

\subsubsection*{WP4: Instrumentation and techniques}
\noindent We will develop novel instrumentation needed for many of the studies listed above, which largely results from combining a diverse range of probing experiments with tuning parameters such as pressure, strain or magnetic field.
\paragraph{Low-T magnetometry under pressure} 
\paragraph{Strain experiments in high magnetic fields} 
\paragraph{AC calorimetry into 100kbar range} \highlight{Note success in piston-cylinder cells} \cite{squire23}.
\paragraph{Ultrasound} will be used to measure sound velocity and sound attenuation. The sound velocity measures different elastic constants, selected by the polarisation and propagation axes. These are thermodynamic quantities, which like the heat capacity are sensitive to phase transitions, providing a reliable method for mapping out phase diagrams as a function of temperature, pressure and field. The attenuation (at low temperature) measures the electronic density of states but is also sensitive to defects in particular the coupling of latter to different order parameters. The different sources of attenuation can be distinguished by measuring at different sound frequencies. Ultrasound was one of the key experimental probes that validated the BCS theory as described in the original 1957 paper. The version we employ takes this to a new level harnessing advanced ultrafast electronics developed for modern telecommunications. %Our work has shown that as might be expected 
Ultrasound is particularly sensitive for detecting CDW order as well as SC.

A planned extension of this work will be to look for acoustic quantum oscillations at high magnetic field \cite{shoenberg84,yoshizawa00,suslov06}.  %These are described in Shoenberg’s book, but have to date received little attention in strongly correlated metals e.g. 
%Yoshizawa et al Physics B 281 740 (2000) [also there is an article in URu2Si2 [LOW TEMPERATURE PHYSICS, PTS A AND B, 850, 1173 (2006) that I can’t access]. 
Such oscillations are much less susceptible to harmonic mixing and will help confirm the interpretation of the quantum interference described in WP 2.

Since ultrasound is a directional probe it is ideally suited to determining the presence of order parameter nodes along different crystal directions to determine the symmetry of the order parameter.

\paragraph{Methodology:}
The pulse-echo measurement technique will be used since it works in both piston cylinder and anvil pressure cells (see below), allows measurements on the same sample at different frequencies and is more easily interpreted than resonance techniques.

In our realisation of this technique 1mm disk LiNBO3 transducers are excited with a short (sub microsecond) pulsees of sound at a harmonic overtone of the transducer f>100Mhz. The transducer both generates the sound in the sample and is then switched to listen to the echoes of sound which is successively reflects back-and-forth through the sample. The captured echoes allow very precise measurement of changes of velocity (with ppm resolution) and of the attenuation.

We have refined this set up to successfully measure U6Fe under pressure, covering both the CDW and superconducting states \highlight{(figure)}. In this case the single crystals are 3mm long and the transducer is attached directly to a polished sample. For UTe2 the current best crystals have dimensions of less than 1mm. To measure sub mm crystals we use a sapphire buffer rod to temporarily separate the detection and generation of sound. We have tested this successfully with a sound frequency of 353 MHz on UTe2. For Indium we have also demonstrated that we can measure the change in attenuation due to superconductivity in samples as thin as 20 microns (figure inset); this demonstrates that the method can be used on small samples in a diamond or sapphire anvil cell.

We will study different quality crystals of UTe2.  The work will also be expanded to look at other compounds synthesisised in WPxxx. 
\paragraph{Miniature piston-cylinder and anvil cells for rotation studies}

%\vspace{-1em}
\subsection*{Plan of work, management,
risks}
\noindent
The experimental, theoretical and computational expertise of numerous UK and international partners complements our strengths in materials growth, exploration and discovery as well as high pressure, high magnetic field measurements. 
%generate new insights into the origins of superconductivity and non-Fermi liquid states 
%that will assist the search for new superconducting material families, ultimately with superior properties for practical applications.
%in Fe-based materials, moderate and ultra-heavy Kondo-lattice materials and quasiperiodic host-guest structures.

% and related transition metal
%compounds more generally, 
%We also expect new insights on the origin of the wide variation of $T_c$ within material families such as the iron-based superconductors. 
%These insights will assist the search for entirely new superconductors, ultimately with superior properties for practical applications. % such as elevated $T_c$, critical field, improved metallurgy or simply lower manufacturing cost.

%Resolving these questions in YFe$_2$Ge$_2$ may furthermore shed new light on Ce-based superconductors such as CeCu$_2$Si$_2$, which have recently come under renewed scrutiny \cite{yamashita17} \hl{(link to next two sections)}.

\paragraph{Plan of work:} the attached chart outlines the project schedule, which is organised along the three work packages 
%(WP 1) in-house measurements (Grosche, Sutherland, Worasaran, Alireza), (WP 2) collaborative measurements  (Chen, Grosche) with associated theory (Chubukov) and numerical studies (Monserrat, Pickard), and (WP 3) crystal growth and materials discovery (Chen). Analysis and interpretation accompanying these activities will be coordinated by Grosche and Lonzarich.  We will schedule in-house measurements according to urgency and sample availability. We will start with CeSb$_2$ and  high pressure Sb-II, to be followed by the iron-based superconductors, then CeNi$_2$Ge$_2$/CePd$_2$Si$_2$, and then materials requiring higher pressures, such as Ba-IV or Ca$_2$RuO$_4$. Collaborative measurements follow the scheduling of our project partners, some of whom have already initiated exploratory studies.

\paragraph{Management:} 
%Responsibilities within the core team are clearly separated, as listed above. 
%the core team is located in the same laboratory. %its work can be coordinated by weekly group meetings. 
Selection of materials, contingency planning and new
opportunities will be decided during weekly group meetings or,
in case of urgency, at additional impromptu meetings.  Collaborative
work with multiple project partners can carry on in parallel and will
be coordinated via long-distance communications. 
%Some projects, in particular cuprate crystal growth, require visits to Cambridge by Bristol collaborators, whereas others 
Visits to collaborating groups will be
prepared by the investigators concerned and finalised in the weekly 
meetings.

\paragraph{Risks and rewards:} we have carefully considered the risks and
rewards of our ambitious proposal and conclude that they are
adequately balanced. Risks are mitigated by (i) the spread of
projects, which range from immediately achievable to extremely
challenging, (ii) our combined experience over many years of research
and the state-of-the-art capabilities of our facilities, (iii) the large
and expanding pool of materials that can be investigated, (iv) the
great diversity of quantum phenomena of theoretical and practical
interest that are expected to arise beyond those discussed above.

%\vspace{-0.5em}


\subsection*{National importance}
\paragraph{Societal and economic impact:}
the superhydride discoveries show that the technological benefits of superconductivity are not fundamentally limited to low temperatures. 
New superconducting materials with superior properties, be it transition temperature,
critical magnetic field, metallurgy or cost, 
can unlock transformative impact, often with particular relevance to sustainability or health: 
(i) powerful magnets already used in MRI scanners, % (a multi-billion pound market \cite{stfc16}), 
\myul{fusion research}  (ITER and private enterprises Commonwealth Fusion Systems and Tokamak Energy), and accelerators (LHC), requiring  thousands of tons of high critical field superconducting wire; 
(ii) lightweight generators already used in wind turbines and motors/generators now examined for use in airplanes;   
(iii) radio-frequency and microwave devices such as exceedingly sharp, low-noise filters for base stations of radio communications systems; 
(iv) ultra-fast, ultra-low-power electronics with applications in communications and computing, where traditional electronics is reaching its performance limits; 
(v) solid-state based quantum computing such as Google's ``quantum supremacy" breakthrough.

%In this project, we investigate the fundamental science underlying unconventional superconductivity in Fe and Cu based materials. 
The project will prepare the ground for a systematic exploration of new unconventional superconductors, and likely serendipitous discoveries carry the potential for entirely unanticipated new technologies. 
% Supporting our research plan places the UK at the heart
%of innovation in the field.
%Successful UK high technology enterprises such as Oxford
%Instruments, Cryogenic, IceOxford have the size, expertise and
%customer-base to implement advances in refrigeration technology. 
Further impact arises from the advanced training our graduate students and PDRAs receive in condensed matter physics and methodology. This
work contributes to the UK effort in a key scientific area and feeds new materials and techniques as well as skilled problem-solvers and
entrepreneurs into our emerging network of high technology instrument
makers.


\paragraph{Academic beneficiaries:}
This project contributes to the \uline{strong UK research in quantum materials}. It connects with work on cuprate and iron-based superconductors in \myul{Bristol and Oxford}, uranium-based superconductors and high pressure research in \myul{Edinburgh}, non-centrosymmetric superconductors, organic superconductors and topological materials in \myul{Warwick}, ruthenates and other 2D materials in \myul{St. Andrews} and \myul{Birmingham}, and Yb-based superconductors at \myul{RHUL}, with theory work at \ul{Bristol, Oxford, Kent, Loughborough, Birmingham, KCL, RHUL, UCL and Cambridge}, and with numerous other quantum materials research initiatives throughout the UK.   
%The UK condensed matter research community, in which the proposed research is embedded, is vibrant and successful. 
%To maintain a leading position within
%a very strong international field, and 
Motivated by  
the high scientific and economic impact of quantum materials research, leading industrial nations have
invested heavily in this field, notably the USA, China, Japan and the
other large European countries. 
To ensure that the UK 
can benefit from any breakthroughs and know-how arising, we must push
forward with ambitious research programmes which leverage existing
strengths. The project falls within the EPSRC research areas {\em Condensed matter: electronic structure} and  {\em magnetism and magnetic materials} as well as {\em Superconductivity}, and within the EPSRC themes {\em Physical Sciences}  and {\em Energy}. It is relevant to the Physics Grand Challenges {\em Emergence and physics far from equilibrium} and 
{\em Quantum physics for new quantum technologies.} 
%Bristol (Carrington, Hussey, Hayden, Friedemann, Annett) and Oxford  (A. Coldea, R. Coldea, Boothroyd), uranium-based superconductors in Edinburgh (Huxley), non-centrosymmetric superconductors, organic superconductors and topological materials in Warwick (Balakrishnan, Lees, Goddard, Petrenko), as well as theory work at Kent (Quintanilla), Loughborough (Betouras), Birmingham (Schofield), KCL (Bhaseen), RHUL ( and UCL (Green)

%The diversity of electronic states in quantum materials, their reach into practicable temperature regions and their tunability can lead to new technologies. 
%Foremost among these is superconductivity, a macroscopic quantum phenomenon with multiple applications: 
%The recent discovery of superconductivity near room temperature in high pressure LaH10 demonstrates that cryogenic temperatures are not a fundamental prerequisite for superconductivity, if materials with sufficiently strong attractive electronic interactions can be identified. 
%For next-generation superconducting magnet technology -- needed, for instance, in MRI magnets, highly efficient generators and particle accelerators -- cuprate superconductors offer step improvements in peak field and operating temperature, whereas  Fe-based superconductors may have the advantage in terms of manufacturing cost. 

%Can other technologically relevant parameters be likewise improved, given the right materials?








\vspace{-1.0em}
%%% Local Variables: 
%%% mode: latex
%%% TeX-master: "Case"
%%% End: 


